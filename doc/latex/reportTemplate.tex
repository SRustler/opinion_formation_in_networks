\documentclass[11pt]{article}
\usepackage{geometry}                
\geometry{letterpaper}                   

\usepackage{graphicx}
\usepackage{amssymb}
\usepackage{epstopdf}
\usepackage{natbib}
\usepackage{amssymb, amsmath}
\DeclareGraphicsRule{.tif}{png}{.png}{`convert #1 `dirname #1`/`basename #1 .tif`.png}

%\title{Coevolution of Non-spectral Opinions in Random Networks}
%\author{Fabian Rußmann, Stefan Rustler}
%\date{date} 

\begin{document}



\thispagestyle{empty}

\begin{center}
\includegraphics[width=5cm]{ETHlogo.eps}

\bigskip


\bigskip


\bigskip


\LARGE{ 	Lecture with Computer Exercises:\\ }
\LARGE{ Modelling and Simulating Social Systems with MATLAB\\}

\bigskip

\bigskip

\small{Project Report}\\

\bigskip

\bigskip

\bigskip

\bigskip


\begin{tabular}{|c|}
\hline
\\
\textbf{\LARGE{Dynamics of Religious Views in Networks}}\\
\\
\hline
\end{tabular}
\bigskip

\bigskip

\bigskip



\LARGE{Fabian Rußmann \& Stefan Rustler}


Group Name: Social State Physicists

\bigskip

\bigskip

\bigskip

\bigskip

\bigskip

\bigskip

\bigskip

\bigskip

Zurich, Switzerland\\
December 20012\\

\end{center}



\newpage

%%%%%%%%%%%%%%%%%%%%%%%%%%%%%%%%%%%%%%%%%%%%%%%%%

\newpage
\section*{Agreement for free-download}
\bigskip


\bigskip


\large We hereby agree to make our source code for this project freely available for download from the web pages of the SOMS chair. Furthermore, we assure that all source code is written by ourselves and is not violating any copyright restrictions.

\begin{center}

\bigskip


\bigskip


\begin{tabular}{@{}p{3.3cm}@{}p{6cm}@{}@{}p{6cm}@{}}
\begin{minipage}{3cm}

\end{minipage}
&
\begin{minipage}{6cm}
\vspace{2mm} \large Name 1

 \vspace{\baselineskip}

\end{minipage}
&
\begin{minipage}{6cm}

\large Name 2

\end{minipage}
\end{tabular}


\end{center}
\newpage

%%%%%%%%%%%%%%%%%%%%%%%%%%%%%%%%%%%%%%%



% IMPORTANT
% you MUST include the ETH declaration of originality here; it is available for download on the course website or at http://www.ethz.ch/faculty/exams/plagiarism/index_EN; it can be printed as pdf and should be filled out in handwriting


%%%%%%%%%% Table of content %%%%%%%%%%%%%%%%%

\tableofcontents

\newpage

%%%%%%%%%%%%%%%%%%%%%%%%%%%%%%%%%%%%%%%



\section{Abstract}

\section{Individual contributions}

The majority of the report was created in a cooperative manner.

\section{Introduction}

\textit{Start with general questions that this work could help to answer
Explain and extend notion of opinion, opinion holder, network and interaction.
Introduce specific example that this work compares its results to.}

We want to study the mechanisms of opinion formation in a network of people. In addition, we also allow the network itself to be adaptable to the opinions existing on it, making two interdependent forces of network evolution and opinion formation measurable.

The motivation for this project involves two different angles on very fundamental dynamics of our society. First of all, we would like to understand the ways in which humans become who they are under the influence of their environment. How do people form their opinions, values, and beliefs and how do their friends and acquaintances play a role in this? Secondly, we are interested in the way our networks of friends and social ties form in the first place. How and why do we choose to be friends with certain people and not with others? How do networks of people in a society form? 

From an intuitive point of view (considering that we are all social beings) most people would argue that the two aspects are interdependent or even that they are two extremes of the same process: Our social environment certainly shapes what we believe and which opinions we hold, while in turn our own values and opinions influence whom we choose to connect with and make part of our social network. On the basis of this rather vague but plausible assumption, our project is an attempt to disentangle and study the effects of these two mechanisms by the means of an abstract, quantitative model. 

An example of a system that one would expect to be subject to such behavior and that we would like to put a focus on is religious affiliation within a society. It could be argued that the social surroundings have an effect on (or, as an extreme, completely determine) which religion a person chooses to belong to. Also, one's religion also influences to whom we connect with socially, for example through the community in a church (the opposing extreme would be that a person's social ties are entirely composed of member's of the same religion). 



\section{The Model}

In this work the network of opinion holders will be modeled by means of a graph with $N$ nodes and $M$ edges connecting the nodes. Thus "graph" in this context is a collection of edges and nodes and not to be confused with graphs of functions. 

Each node represents an opinion holder, e.g. a  single person, indexed by $i$, to whom a certain opinion $g_i$ is randomly assigned. The number of all possible opinions $G$ is only limited by the number of nodes existing in the network, i.e. $N$, to a certain factor $\gamma = \frac{N}{G}$, where $\gamma > 1$. In this sense, it is not possible for every single individual to hold a unique opinion. In this model we are externally setting $\gamma$, i.e. the initial average number of nodes with the same opinion, as opposed to the number of existing opinions. 

It is important to point out the nature of opinions this model assumes. Though different opinions will be denoted by a set of subsequent numbers given by $G$, this is not to be mistaken with a spectrum of opinions with two extremes. That is no opinion is more similar to another than yet another. Each opinion is equally likely to be adopted by an opinion holder. This will have important implications on the type of empirical data the results of this model can be compared to.

The edges that connect two nodes represent a bridge of interaction of two nodes, i.e. direct interaction can only occur between two connected nodes. Per definition an individual edge is directed, which means that information can only be carried from one node to another along its direction. In our model each connection will be undirected, i.e. two nodes can influence \textit{each other}. This manifests itself in the fact that only antiparallel \textit{double}-edges can occur. Furthermore, we have forbidden self-edges and multi-edges to occur. In this sense a node as such is already connected to itself and can also not connect with certain nodes more strongly than with others via more edges. The number of edges leaving a node $i$ is given by the degree $k_i$. The average degree of the graph is therefore given by $M=\frac{kN}{2}$. By externally setting $N$ and $k$, edges are distributed in a random and uniform fashion: each pair of nodes is equally likely to be connected by an edge. This random graph generation was first proposed by Paul Erdős and Alfréd Rényi.\cite{Erdos} There are many other types of graphs but due its simplicity and wide applicability, this work focusses on random graphs of different sizes. 

Even though we are using the interpretation of nodes and edges as people with opinions and social connections for reasons of intuition, it is important to not restrict the notion of connected and interacting nodes to such a specific context. The concept of nodes holding and influencing each other's opinion is much broader as mentioned in the introduction.

The above-mentioned setup of nodes, edges and opinions is now subject to time-evolution. Specifically, nodes are allowed to interact such that a \textit{co}evolution of edges and opinions occurs: 

In two basic update steps, we will enable each node to follow one of the two mechanisms, reconnecting to a node of equal opinion \textit{or} adapting his/her opinion to the neighborhood. A tunable probability $\Phi$ of choosing either one of them will be included in the model. In this work this probability will be referred to as \textit{reconnection probability}. In more detail, at each time step the following will be done:

\begin{enumerate}
\item Pick a random node $i$ with opinion $g_i$. 
\item If $k_i=0$, do nothing. 
\item Otherwise, with probability $\Phi$ randomly select one of the nodes $j$ that $i$ is connected to.

a. If $g_j \neq g_i$ choose a different $ij$-pair, i.e. start over at step 1.

b. Otherwise, connect to $j$ of same opinion, i.e. $g_j = g_i$.

\item Otherwise, with probability $1-\Phi$ randomly select one of the neighboring vertices $j$ and change $g_i$ to $g_j$.
\end{enumerate}

Here the third step corresponds to the mechanism in the network of an individual reconnecting to like-minded individuals, whereas the fourth step to that of an individual adapting his/her opinion to his/her neighborhood. 

Steps 1. to 4. are iteratively performed until a convergent or equilibrium state is achieved. The attentive reader might have noticed in the steps above that step 3a could result in an infinite loop as soon as every node is only connected to nodes of equal opinion. This is exactly the definition of the convergent state and will be referred to as the \textit{consensus state}. We will be mainly concerned with how this convergent state looks like. Separated clusters of differen opinions will have formed, whose size distribution in dependence of several external parameters, namely $\Phi$, $N$ and $k$, this work is mainly concerned with.

Intuitively, one can picture two extreme scenarios: one in which there is one prominent opinion, and one in which there is no such opinion but several much less prominent ones. In other words, an emergent phenomenon of different manifestation is to expected as macrocopic outcome from the microscopic steps desecribed above.

\section{Theory of Critical Phenomena}

Several analogies can be drawn from the system of interacting nodes to systems in solid state physics. Therefore some formulas and concepts of this field will be applied in this work, all of which will be derived and explained in this section. As indicated in the previous section, two extreme scenarios are possible. Speaking in physics terms, two phases that can transition into each other are expected to occur. An individual phase is characterized by a several internal parameters, most importantly the so-called \textit{order parameter}. If this parameter at some point changes drastically upon slight variation of another external parameter, one can speak of a \textit{phase transition}. This point of drastic change is called the \textit{critical point}. A simple example is the freezing of water. Upon slight variation of the (external) temperature near the freezing point, a drastic change of (internal) density is observed. In this work the external variable will be the reconnenction probability $\Phi$, whereas the internal order parameter will be the size of the biggest opinion cluster, denoted by $S$. The critical point at which a phase transition occurs will be denoted by $\Phi_c$.

Behavior near the critical point. Critical exponents, reduced

\begin{equation}
\varphi = \Phi - \Phi_c
\end{equation}

\begin{equation}
S(\varphi) \sim  |\varphi|^\alpha
\end{equation}

Scaling law

generalized homogeneous function of the free energy. Consensus state is state of low energy.

\begin{equation}
F(\lambda \varphi) = \lambda F(\varphi)
\end{equation}

$\lambda = N^d$

\begin{equation}
f(N^a \varphi) = N^d f(\varphi)
\end{equation}

We do not know how it looks like but we can imply that S is a derivative of it. Can we?

\begin{equation}
S(\varphi) = N^{an!-d} S(\varphi)
\end{equation}

\begin{equation}
S(\varphi) = N^{-a}\cdot S(N^b  \varphi)
\end{equation}

\section{Implementation}

\subsection{Main Script}

Add pseudo code

\subsection{Simulation Script}

Add pseudo code

\subsection{Data Plotting}

\section{Results and Discussion}

In the following $k = 4$ 
100 runs
high-perfomance cluster of ETH Zürich, $Brutus$,


\subsection{Cluster Size Distribution}

Qualitative difference below and above critical $\Phi$

Explain plotting and graphs in detail.

\subsection{Renormalization and Critical Exponent}

Determine $\Phi_c$ from graphs of different $N$. Renormalize them such that critical exponent can be extracted. Discuss similarity





\subsection{Comparison with Empirical Data}

As an optional task one can compare the results, i.e. the opinion size distribution, with actual data on opinions of a certain aspect, e.g. religious view. One could then think about by what factors our $\Phi$ is influenced in reality.

Find meaningful distributions of religious views.

For the optional comparison with empirical data, we would, depending on what the data looks like, intend to use common methods of statistical analysis, e.g. OLS regression, fitting functions on linear and/or logarithmic scales and comparing the values of their coefficients with the fitted distribution functions obtained by our simulations.

\section{Summary and Outlook}

\section{Appendix}

\subsection{References}
\bibliography{reportTemplate}
Erdős, Paul; A. Rényi (1960). "On the evolution of random graphs". Publications of the Mathematical Institute of the Hungarian Academy of Sciences 5: 17–61.

\subsection{List of Figures}

\subsection{List of Tables}

\subsection{Source Code}






\end{document}  



 
